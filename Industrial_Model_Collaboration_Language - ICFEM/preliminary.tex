\section{Industrial Modelling Collaboration Language (IMCL)}
% In this section, we will introduce the \emph{Industrial Modelling Collaboration Language}(IMCL), which can describe the physical resources and systems in an unified model. This language achieves platform independence of different systems, and supports the decomposition and collaboration of the complex system.
% The \emph{IMCL} is an event-triggered imperative language, which including the composition of event triggered, message communication and data synchronism under the constraint of system resources.
% The \emph{IMCL} is an event-triggered imperative language, which including the composition of event trigger, message communication and data synchronism under the constraint of system resources.
\subsection{The Syntax of $IMCL$}

The abstract syntactic sets of \emph{} is defined as follows:
% 定义数值
\begin{equation*}
    \begin{aligned}
        A_{exp} ::= &\ val \mid X \mid a_{0} + a_{1} \mid a_{0} - a_{1} \mid a_{0} * a_{1} \mid a_{0} / a_{1} \\
        B_{exp} ::= &\ \top \mid \bot \mid a_{0} = a_{1} \mid a_{0} \neq a_{1} \mid a_{0} > a_{1} \mid a_{0} < a_{1} \\
        C_{exp} ::= &\ channel!m \mid channel?m \mid sync.n \\
        E_{exp} ::= &\ E_{0};E_{1} \mid E_{0} \triangleleft b \triangleright E_{1} \mid b * E \mid X := a \mid C_{exp} \mid a \gg Dev \mid a \ll Dev \\
        T_{exp} ::= &\ trigger \ b \diamond E \mid trigger \ channel?m \diamond E  \\
    \end{aligned}
\end{equation*}

%\begin{itemize}
%  \item $A_{exp}$ is the arithmetic expression. Where $val$ is value, the $X$ a variable and $a_{0}, a_{1} \in A_{exp}$.
%  \item $B_{exp}$ is the boolean expression. Where $\top$ and $\bot$ are signifying the \emph{true} and \emph{false} respectively. For every \emph{b} $\in$ \emph{Bexp}, it is a bool expression.
%  \item $C_{exp}$ is the communication expression. The $channel!m$ signifies sending a value $m$ from a $channel$ . The $channel?m$ signifies receiving a value $m$ from a $channel$. The $sync.n$ means that get the synchronous data $n$.
%  \item $E_{exp}$ is the execution expression. For $E_{0},E_{1} \in Eexp$. $E_{0};E_{1}$ represents the sequential composition of $E_{0}$ and $E_{1}$. The syntax of condition choice $E_{0} \triangleleft b \triangleright E_{1}$ behaves like $E_{0}$ if $b$ is true, or $E_{1}$ otherwise. The $b * E$ indicates that the $E$ will iterates until $b$ is false. Assignment expression $X := a$ assigns the value of the expression a to the variable X. The $C$ is a $C_{exp}$ which indicates that it can be denoted as an $E_{exp}$. Specifically, for a value $a$, the $a \gg Dev$ denotes that system transmit the $a$ to the the physical device $Dev$ in a system, and $Dev \ll a$ indicates that getting the $a$ from $Dev$.
%  \item $T_{exp}$ is the event-trigger expression. The $trigger \ b \diamond E$ denotes an event-trigger happens when the $b$ is true. The $ trigger \ channel?m \diamond E_{exp}$ indicates that when receiving a $m$ from the $channel$, the event $E$ will begin to execute.
%\end{itemize}
$A_{exp}$ is the arithmetic expression where $val$ is value, $X$ is variable, $a_{0}, a_{1} \in A_{exp}$.
$B_{exp}$ is the boolean expression, where $\top$ and $\bot$ are signifying the \emph{true} and \emph{false}. For every \emph{b} $\in$ $B_{exp}$, it is a bool expression. Next, we mainly introduce some specially expression in \emph{IMCL}. $C_{exp}$ is the communication expression. The $channel!m$ signifies sending a value $m$ from a $channel$ . The $channel?m$ signifies receiving a value $m$ from a $channel$. The $sync.n$ means to get the synchronous data $n$.
$E_{exp}$ is the execution expression. Condition choice $E_{0} \triangleleft b \triangleright E_{1}$ is that if $b$ is true then $E_{0}$, else $E_{1}$. The $b * E$ indicates that the $E$ will iterates until $b$ is false. Specifically, $a \gg Dev$ denotes that system transmit value $a$ to physical device $Dev$, while $Dev \ll a$ is getting $a$ from $Dev$. $T_{exp}$ is the event-triggered expression. The $trigger \ b \diamond E$ denotes an event $E$ happens when the $b$ is true. The $ trigger \ channel?m \diamond E_{exp}$ indicates that when receiving a $m$ from the $channel$, the event $E$ will begin to execute.

%\subsection{Modelling the Industrial System Using IMCL}
\subsection{\emph{IMCL} Model for Industrial Control System}

The modelling process focuses on system functions and features as following two steps:
%
%\begin{enumerate}
%  \item \textbf{Modelling the physical resources} \\
%  Representations of physical resources are different based on diverse industrial environment, for instant, sensors, read-write devices and the other resources. Considering their effects on the whole system, we abstract describe all those resources as variables to unifying definition of resources.
%  \item \textbf{Modelling the system} \\
%  Observing its behavior, the nature of system is gathering functions, read-write operations, and other actions together. Similar to physical resources, we model them as execution expressions. Multiple execution expressions in one specific order can make up one trigger event marked as $T$.
%
%  An IMCL model consists of multiple concurrent trigger events, and every event is triggered only when its condition is satisfied. Then, the IMCL model can be defined:
%    \begin{displaymath}
%    Prog = \overset{n}{\underset{i=1}{\bowtie}} T_{i}, n \in N^{+}
%    \end{displaymath}
%    Where the \emph{Prog} denotes one system model, \emph{$T_{i} \in T_{exp}$} demotes the event-triggered expression int system model. The $\overset{n}{\underset{i=1}{\bowtie}} T_{i}$ signifies those concurrent event-triggers are concurrent in one system. For instrant, a given $Prog = T_{1} \bowtie T_{2} \bowtie T_{3}$, which demotes that the model $Prog$ contains three trigger events: $T_{1}$, $T_{2}$ and $T_{3}$. For different IMCL models, they can interact with each other through message communication and data synchronization.
%\end{enumerate}


\textbf{Modelling the physical resources. }
Representations of physical resources are different based on diverse industrial environment, for instance, sensors, read-write devices, and the other resources. Considering their effects on the whole system, we describe all those resources as variables to unifying definition of resources.

\textbf{Modelling the system. }
Observing its behavior, the nature of the system is gathering functions, read-write operations, and other actions together. Similar to physical resources, we model them as execution expressions. Multiple execution expressions in one specific order can make up one trigger event marked as $T$.

An \emph{IMCL} model consists of multiple concurrent trigger events, and every event is triggered only when its condition is satisfied. Let $\bowtie$ be the concurrent operation of two events s.t. $T_1\bowtie T_2$. Then, the \emph{IMCL} model \emph{Prog} can be defined:
\begin{displaymath}
Prog = \overset{n}{\underset{i=1}{\bowtie}} T_{i}, n \in N^{+}
\end{displaymath}
% Taking $Prog = T_{1} \bowtie T_{2} \bowtie T_{3}$ for example, the model $Prog$ contains three trigger events: $T_{1}$, $T_{2}$ and $T_{3}$.
% For different IMCL models, they can interact with each other through message communication and data synchronization.


\textbf{\emph{Example 1.}} \ For an industrial control system, the data collector \emph{collects} information and saves it to the \emph{database}. When the temperature \emph{sensor} is detected above 200 degrees, the system turns on the \emph{fan} to cool down. The \emph{IMCL} model of this system is shown as follow:

% \tiny\tiny\scriptsize\footnotesize\small\normalsize\large\Large\LARGE\huge\Huge\footnotesize\small\normalsize\large\Large\LARGE\huge\Huge
\definecolor{mygreen}{rgb}{0,0.6,0}
\definecolor{mygray}{rgb}{0.5,0.5,0.5}
\definecolor{mymauve}{rgb}{0.58,0,0.82}
\linespread{0.65}
\lstset{ %
backgroundcolor=\color{white},   % choose the background color
basicstyle=\scriptsize\ttfamily,        % size of fonts used for the code
 columns=fullflexible,         % 设置空格填充
breaklines=true,                 % automatic line breaking only at whitespace
captionpos=b,                    % sets the caption-position to bottom
tabsize=4,
commentstyle=\color{mygreen},    % comment style
escapeinside={\%*}{*)},          % if you want to add LaTeX within your code
keywordstyle=\color{blue},       % keyword style
stringstyle=\color{mymauve}\ttfamily,     % string literal style
frame=single,
numbersep=8pt,  %设置行号与代码的距离,默认是5pt
rulesepcolor=\color{red!20!green!20!blue!20},
% identifierstyle=\color{red},
language=c++,
}
\begin{lstlisting}
  // Physical resources in system
  SENSOR: sensor;
  DEVICE: collector, database, fan;

  program : Example() [VAR: tmp, data]
  {
      TRIGGER(sensor > 200)   // trigger event: T1
      {
          tmp := cool;
          tmp >> Fan;
      }

      TRIGGER(TRUE)           // trigger event: T2
      {
          data << collector;
          data >> database;
      }
  }
\end{lstlisting}


We abstract the physical resources in system as variables. The $T_1$ and $T_2$ are two  trigger events that represent different functions. The system model can be described like $Prog = T_{1} \bowtie T_{2}$.



% 版本2
%Let $val \mid val \in \mathbb{R} \cup \Sigma^{+}$ be values. Where $\Sigma$ denotes the alphabet collection, and $\Sigma^{+}$ means that it is a string type.
%The $Vals$ signifying the static value used in the program. Let $Vars$ be a collection set of variables. The relation: $Vars \to Vals$ records the values of all the variables. %\textbf{X} and \textbf{Y} range over $Var$.
%The $val$ is a value ranging over $Vals$, $X$ is a variable ranging over $Vars$, and $a_{0}$ and $a_{1}$ are both the arithmetic expressions.
%
%Where $A_{exp}$ is the definition of arithmetic expression. The $a_{0} + a_{1}$, $a_{0} - a_{1}$, $a_{0} + a_{1}$ and $a_{0} + a_{1}$ indicate the four rules of arithmetic $add$, $subtract$, $multiply$ and $divide$, respectively.
%
%Where $B_{exp}$ is the definition of boolean expression. $\top$ and $\bot$ are signifying the \emph{true} and \emph{false} respectively.
%$a_{0}=a_{1}$ denotes that $a_{0}$ is equal to $a_{1}$, and $a_{0} \neq a_{1}$ means that $a_{0}$ is not equal to $a_{1}$.
%$a_{0} > a_{1}$ indicates that $a_{0}$ is bigger than $a_{1}$ and $a_{0} < a_{1}$ means that $a_{0}$ is smaller than $a_{1}$. For every \emph{b} $\in$ \emph{Bexp}, it is a bool expressions.
%
%Where $C_{exp}$ is the definition of communication expression. The $channel!m$ in process signifying sending a value $m$ from a $channel$ when the $channel$ is empty. The $channel?m$ in process signifying receiving ing a value $m$ from a $channel$ when the $channel$ is not empty. The syntax of $sync.n$ means that get the synchronous data $n$.
%
%Where $E_{exp}$ is the definition of execution expression. $E_{0},E_{1} \in Eexp$  $E_{0};E_{1}$ represents the sequential composition of $E_{0}$ and $E_{1}$. The syntax of condition choice $E_{0} \triangleleft b \triangleright E_{1}$ behaves like $E_{0}$ if b is true, or $E_{1}$ otherwise. The $b * E$ indicates that the $E$ will iterates until $b$ is false. Assignment expression $X := a$ assigns the value of the expression a to the variable X. The $C$ is a $C_{exp}$ which indicates that it can be denoted as an $E_{exp}$.
%Specifically, the $a \gg Dev$ denotes that system transmit the $a$ to the $Dev$ which is the device in a system, and $Dev \ll a$ indicates that getting the $a$ from $Dev$.
%
%Where $T_{exp}$ is the definition of event-trigger expression. The $trigger \ b \diamond E$ denotes an event-trigger happens when the $b$ is true. The $ trigger \ channel?m \diamond E_{exp}$ indicates that when receiving a $m$ from the $channel$, the event $E$ will begin to execute.
%
%For every IMCL program, it is a set of multiple concurrent event-triggers, which can be defined as follow:
%
%\begin{displaymath}
%    Prog = \overset{n}{\underset{i=1}{\bowtie}} T_{i}, n \in N^{+}
%\end{displaymath}
%
%The \emph{IMCL} can be used to model the industrial control system.  $\overset{n}{\underset{i=1}{\bowtie}} T_{i}$ signifying the relation of some concurrent event-triggers in one system. For example, a given $Prog = T_{1} \bowtie T_{2} \bowtie T_{3}$, which demotes that the model $Prog$ contains three trigger events: $T_{1}$, $T_{2}$ and $T_{3}$. For different IMCL models, they can interact with each other through message communication and data synchronization.




% 版本1
% \textcolor[rgb]{1,0,0}{.....}
%\subsection{Program Analysis}

%\textbf{Program Analysis}In computer science, program analysis is the process of automatically analyzing the behavior of computer programs regarding a property such as correctness, robustness, safety and liveness. In general, program analysis focuses on two major areas: program optimization and program correctness. The first focuses on improving the program’s performance while reducing the resource usage while the latter focuses on ensuring that the program does what it is supposed to do.
%
%Program can be roughly divided into two categories:1) static analysis which is performed without exectuting the program; 2) dynamic analysis which is performed during runtime. But in practice, an analysis may be performed in a combination of both.
%
%Static analysis usually be conducted upon the program source code, intermediate code and object code. It used to be consisted of four effective core methods including Controal-Flow Analysis, Data-Flow Analysis, Abstraction Interpretation and Type and Effect Systems. Nowadays, the method of model checking is introduced for program verfication.
%
%Dynamic anlysis is based on runtime knowledge of the program, which can be used to increasing the precision of the analysis while also providing runtime protection. Howerver, dynamic analysis is imcomplete and invasive, beacuse it can only analyze a few single executions of the program and might degrade the program's performance due to runtime checks.


%\textbf{Program Slicing} For a given subset of a program’s behavior, program slicing consists of reducing the program to the minimum form that still produces the selected behavior. The reduced program is called a “slice” and is a faithful representation of the original program within the domain of the specified behavior subset. Generally, finding a slice is an unsolvable problem, but by specifying the target behavior subset by the values of a set of variables, it is possible to obtain approximate slices using a data-flow algorithm. These slices are usually used by developers during debugging to locate the source of errors.
%
%Based on the original definition of Weiser, informally, a static program slice S consists of all statements in program P that may affect the value of variable v at some point p. The slice is defined for a slicing criterion C=(x,V), where x is a statement in program P and V is a subset of variables in P. A static slice includes all the statements that affect variable v for a set of all possible inputs at the point of interest (i.e., at the statement x)[clarification needed]. Static slices are computed by finding consecutive sets of indirectly relevant statements, according to data and control dependencies.



% 语法分节定义
%\section{Preliminary}
%
%In this section, we will introduce the \emph{Industrial Modelling Collaboration Language}(IMCL), which supports the decomposition and collaboration of the industrial control system. The \emph{IMCL} is an event-triggered imperative language, which including the composition of event triggered, message communication and data synchronism under the constraint of system resources. In addition, this section also introduces the theory of \emph{program analysis}, which is necessary to understand when we want to decompose and collaborate an industrial system model.
%
%\subsection{The Syntax of $IMCL$}
%The abstract syntactic sets of \emph{IMCL} is defined as follow:
%
%% 定义数值
%\begin{definition}[Vals]
%    Values
%    \begin{equation*}
%        \begin{aligned}
%            Vals = \{val \mid val \in \mathbb{R} \cup \Sigma^{+}\}
%        \end{aligned}
%    \end{equation*}
%\end{definition}
%
%Let $Vals$ be a collection set of different types of values. $\Sigma$ denotes the alphabet collection, and $\Sigma^{+}$ means that it is a string type.
%The $Vals$ signifying the static value used in the program.
%
%% 定义变量%%
%\begin{definition}[Vars]
%    Variables
%\end{definition}
%
%Let $Vars$ be a collection set of variables.
%%\textbf{X} and \textbf{Y} range over $Var$.
%The relation: $Vars \to Vals$records the values of all the variables.
%
%% 定义算数表达式
%\begin{definition}[Aexp]
%    Arithmetic expression
%    \begin{equation*}
%        \begin{aligned}
%            A ::= \ val \mid X \mid a_{0} + a_{1} \mid a_{0} - a_{1} \mid a_{0} * a_{1} \mid a_{0} / a_{1}
%        \end{aligned}
%    \end{equation*}
%\end{definition}
%
%Where $A$ is an arithmetic expression. The $val$ is a value that ranges over $Vals$, and $X$ is a variable ranging over $Vars$. $a_{0} + a_{1}$, $a_{0} - a_{1}$, $a_{0} + a_{1}$ and $a_{0} + a_{1}$ indicate the four rules of arithmetic $add$, $subtract$, $multiply$ and $divide$, respectively.
%
%% 定义布尔表达式
%\begin{definition}[Bexp]
%    Boolean expression
%\begin{equation*}
%        \begin{aligned}
%            B ::= \ \top \mid \bot \mid a_{0} = a_{1} \mid a_{0} \neq a_{1} \mid a_{0} > a_{1} \mid a_{0} < a_{1}
%        \end{aligned}
%    \end{equation*}
%\end{definition}
%
%Where $B$ is a boolean expression. $a_{0}$ and $a_{1}$ are both the object of $Aexp$.
%$\top$ and $\bot$ are signifying the \emph{true} and \emph{false} respectively.
%$a_{0}=a_{1}$ denotes that $a_{0}$ is equal to $a_{1}$, and $a_{0} \neq a_{1}$ means that $a_{0}$ is not equal to $a_{1}$.
%$a_{0} > a_{1}$ indicates that $a_{0}$ is bigger than $a_{1}$ and $a_{0} < a_{1}$ means that $a_{0}$ is smaller than $a_{1}$.
%
%% 定义通信表达式
%\begin{definition}[Cexp]
%    Communication expression
%    \begin{equation*}
%        \begin{aligned}
%            C ::= &\ channel!m \mid channel?m \mid sync.n
%        \end{aligned}
%    \end{equation*}
%\end{definition}
%
%Where $C$ is a communication expression. The $channel!m$ in process signifying sending a value $m$ from a $channel$ when the $channel$ is empty. The $channel?m$ in process signifying receiving ing a value $m$ from a $channel$ when the $channel$ is not empty. The syntax of $sync.n$ means that get the
%synchronous data $n$.
%
%% 定义执行集合
%\begin{definition}[Eexp]
%    Execution expression
%    \begin{equation*}
%        \begin{aligned}
%            E ::= &\ E_{0};E_{1} \mid E_{0} \triangleleft b \triangleright E_{1} \mid b * E \mid X := a \mid C \mid a \gg Dev \mid a \ll Dev
%        \end{aligned}
%    \end{equation*}
%\end{definition}
%
%Where $E$ is an execution expression. When $Stop$ in a program occurs, the system will terminate immediately. $E_{0};E_{1}$ represents the sequential composition of $E_{0}$ and $E_{1}$. The syntax of condition choice $E_{0} \triangleleft b \triangleright E_{1}$ behaves like $E_{0}$ if b is true, or $E_{1}$ otherwise. The $b * E$ indicates that the $E$ will iterates until $b$ is false. Assignment expression $X := a$ assigns the value of the expression a to the variable X. The $C$ is a $Cexp$ which indicates that it can be denoted as an $Eexp$.
%Specifically, the $a \gg Dev$ denotes that system transmit the $a$ to the $Dev$ which is the device in a system, and $Dev \ll a$ indicates  that getting the $a$ from $Dev$.
%
%% 定义事件触发
%\begin{definition}[Texp]
%    Event-Trigger expression
%    \begin{equation*}
%        \begin{aligned}
%            T ::= \ trigger \ b \diamond E \mid trigger \ channel?m \diamond E \mid T_{1} \bowtie T_{2}
%        \end{aligned}
%    \end{equation*}
%\end{definition}
%Where $T$ is an event-trigger expression. The $trigger \ b \diamond E$ denotes an event-trigger happens when the $b$ is true. The $ trigger \ channel?m \diamond E$ indicates that when receiving a $m$ from the $channel$, the event $E$ will begin to execute. $T_{1}\bowtie T_{2}$ signifying the relation of two triggers can be concurrent.
%
%
%\subsection{$IMCL$ Model}
%\emph{IMCL} can be used to model the industrial control system. For every IMCL program, it is a set of multiple event-triggers.
%For a given $P_{model} = T_{1} \bowtie T_{2} \bowtie T_{3}$, which demotes that the model $P_{model}$ contains three trigger events: $T_{1}$, $T_{2}$ and $T_{3}$. For different IMCL models, they can interact with each other through message communication and data synchronization.
%

% \textcolor[rgb]{1,0,0}{.....}
%\subsection{Program Analysis}

%\textbf{Program Analysis}In computer science, program analysis is the process of automatically analyzing the behavior of computer programs regarding a property such as correctness, robustness, safety and liveness. In general, program analysis focuses on two major areas: program optimization and program correctness. The first focuses on improving the program’s performance while reducing the resource usage while the latter focuses on ensuring that the program does what it is supposed to do.
%
%Program can be roughly divided into two categories:1) static analysis which is performed without exectuting the program; 2) dynamic analysis which is performed during runtime. But in practice, an analysis may be performed in a combination of both.
%
%Static analysis usually be conducted upon the program source code, intermediate code and object code. It used to be consisted of four effective core methods including Controal-Flow Analysis, Data-Flow Analysis, Abstraction Interpretation and Type and Effect Systems. Nowadays, the method of model checking is introduced for program verfication.
%
%Dynamic anlysis is based on runtime knowledge of the program, which can be used to increasing the precision of the analysis while also providing runtime protection. Howerver, dynamic analysis is imcomplete and invasive, beacuse it can only analyze a few single executions of the program and might degrade the program's performance due to runtime checks.


%\textbf{Program Slicing} For a given subset of a program’s behavior, program slicing consists of reducing the program to the minimum form that still produces the selected behavior. The reduced program is called a “slice” and is a faithful representation of the original program within the domain of the specified behavior subset. Generally, finding a slice is an unsolvable problem, but by specifying the target behavior subset by the values of a set of variables, it is possible to obtain approximate slices using a data-flow algorithm. These slices are usually used by developers during debugging to locate the source of errors.
%
%Based on the original definition of Weiser, informally, a static program slice S consists of all statements in program P that may affect the value of variable v at some point p. The slice is defined for a slicing criterion C=(x,V), where x is a statement in program P and V is a subset of variables in P. A static slice includes all the statements that affect variable v for a set of all possible inputs at the point of interest (i.e., at the statement x)[clarification needed]. Static slices are computed by finding consecutive sets of indirectly relevant statements, according to data and control dependencies.




