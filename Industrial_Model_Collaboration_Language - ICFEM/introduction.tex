\section{Introduction}
``Industry 4.0" is a term originated by the German government for promoting the computerization of manufacturing\cite{industrial4}.
It involves the Cyber-Physical Systems (CPS), the Internet of Things (IoT) and cloudy computing etc in manufacturing. An industrial environment contains the normal work condition of people like offices, physical resources such as manufacturing equipment, storage and so on.
Many wireless networks monitor various aspects of the environment and transfer the data information to a central processor\cite{oliveira2011wireless}. Most of those actual industrial control systems are containing some Computing Units(CUs)\cite{GERO199065}. Every CU has the capability to process tasks and schedule specific physical resources.
However, there are many physical resources in a system. Meanwhile, the industrial control system has a variety of functions and is more intelligent. Hence, the traditional industrial environment has become more complex than before. It leads to the analysis and verification of such  systems more difficult. The cost of system research and application is more expensive.

% ����ϵͳ�����IJ���
To overcome the limitation, decomposition is used to reduce the complexity of the system for research and applications. \textbf{Decomposition} is a technique to master the function complexity of the model. Usually, a functional model can be replaced by a series of models of sub-systems after decomposition.
An actual industrial control system can be treated as a complex model. Every CU in a system can realize some part function of the system and can be considered as the corresponding models of sub-systems. Decomposition is a well-developed technique and it is widely applied to solve the problems brought by the function complexity.

However, most CUs have limits since the physical resource scheduling is limited.  If one system is too complex to be handled by one CU, it needs to use multiple CUs to collaborate with each other.
Many researchers \cite{DBLP:phd/de/Penzenstadler2010,henzinger2001giotto,damavandi2016modeling} have achieved great achievements about the collaboration of industrial system. Many researchers focus on the industrial intelligent ecosystem\cite{s16020215,deGeaFernandez2017} to handle the collection of massive data from the various devices dynamical collaboration.
But they didn't take the complex physical resources into consideration for the whole process of collaboration.
For example, there are three physical resources(\emph{res}1, \emph{res}2, \emph{res}3) in one system and all of those resources should work together, and two CUs($CU_{1}$, $CU_{2}$), where $CU_{1}$ is limited to schedule \emph{res}1 and \emph{res}2, and $CU_{2}$ is limited to schedule \emph{res}1 and \emph{res}3. Neither of the two CUs can implement the system. So, it is necessary to find the best solution to make the two CUs collaboration with each other to implement the system.

In our research, we introduce an event-triggered language called \emph{Industrial Modelling Collaboration Language}(\emph{IMCL}) which is platform-independent for heterogeneous systems. By this language, we can describe the physical resources and systems in one unified model.
Moreover, we propose the decomposition algorithms that can decompose the complicated system model to exact number of sub-systems appropriately corresponding to every CU with the constraints of resources. To maintain its functional consistency between sub-systems and original composited model, we present the collaboration algorithm. Besides, we show the evaluation method to get the optimizing solution of system collaboration.

The main contributions of this paper can be surmised as follows:
\begin{itemize}
  \item We propse an industrial model collaboration language, which can describe the system with complicated physical resources in a unified model.
  \item We provide the decomposition algorithm of the industrial model using \emph{IMCL}. This algorithm mainly includes the program analysis of control flow and data flow and the resource allocation with resource constraints.
  \item We implement the collaboration algorithm with the resource constraint based on the decomposed system. Moreover, we provide the evaluation way to get the most optimal solution for system collaboration.
\end{itemize}

% \subsection{Organization of this paper}

\medskip
\textbf{Outline}
\medskip

The remainder of this paper is organized as follows. In Section 2, we present the abstract of \emph{IMCL}. Section 3 proposes the approach for decomposition and collaboration. In Section 4, we introduce the decomposition algorithm. In Section 5, we discuss the collaboration algorithm. Section 6 shows the case study of one actual industrial control system.
%Section 7 introduces related works about our work. Section 8 closes the paper with a few concluding remarks.
Section 7 is the conclusion.

\medskip
\textbf{Related Work}
\medskip

The problem of the system decomposition and collaboration has much discussed in academic community during the past decades. The industry has been experimenting with different ways to develop new technologies to achieve better solutions for specific industrial environments\cite{korhonen2001four}.
% (Systems integration and collaboration in architecture, engineering, construction, and facilities management: A review)
In \cite{Shen2010196}, it presents that with the rapid advancement of information and communication technologies, particularly the Internet and Web-based technologies during the past years, various systems integration and collaboration technologies have been developed and deployed to different application domains, including architecture, engineering, construction, and facilities management.

% Using program synthesis for program analysis
Oxford University's latest research\cite{david2015using} proves that procedural decomposition and conversion is beneficial to procedural analysis.
% (Weiser,���Է���,���з���)
Weiser et al.\cite{weiser1981program} has proposed the program slicing method, which introduces how to slice a program with the data-flow and control-flow analysis. Jean Bertrand Gauthie\cite{DBLP:journals/orl/GauthierDL14} and Jingde Cheng\cite{Cheng:1993:SCP:646902.710201} have proved the decomposition theorems for linear programs and the approach of slicing concurrent programs, respectively.
The \cite{weiser1981program,DBLP:journals/orl/GauthierDL14,Cheng:1993:SCP:646902.710201} are influential on the basis principle of our decomposition algorithms.
% (Giotto)
Thomas A.Henzinger et al. present a time-triggered language \emph{Giotto}\cite{henzinger2001giotto}, which provides an abstract programmer's model for the implementation of one embedded control system with hard real-time constraints.
\emph{Giotto} can be annotated with platform constraints such as task-to-host mappings, and task and communication schedules. Different from \emph{Giotto}, our research focuses on the way how to describe the system with trigger events.
% (AutoBayes)
The NASA has released a toolkit \emph{AutoBayes}\cite{schumann2008autobayes} that can decompose and convert one system model from the perspective of the data analysis.
% (Better Requirements Decomposition Guidelines Can Improve Cost Estimation of Systems Engineering and Human Systems Integration)
Valerdi et al.\cite{valerdi2010better} details proposed updates to requirements decomposition guidelines that will help generate the number of system requirements.
% (A different view on system decomposition �C subsystem-centered property evaluation in multiple supersystems)
Arne Herberg et al.\cite{DBLP:conf/csdm/HerbergL13} motivates the need for enhanced support for subsystem development and evaluation in large engineering systems.
% (Decomposition of Systems and their Requirements -- Decomposition of Systems and their Requirements)
In \cite{DBLP:phd/de/Penzenstadler2010}, it introduces how to transition from one system to multi-subsystems using a criteria catalog and systematic requirement refinement, and we have inspired a lot from it. In our research, we mainly focus on the constraints of resources, which is the basis rule to decompose one IMCL system.
% (Industrial Internet of Things-Based Collaborative Sensing Intelligence: Framework and Research Challenges)
YuanFang Chen et al.\cite{s16020215} introduce the framework and research challenges about the industrial internet of things-Based collaborative sensing intelligence. Their paper thinks that an industrial intelligent ecosystem enables the collection of massive data from the various devices dynamically collaborating with humans. But in our work, we focus on the collaboration between computing units in the industrial control system.

% (Multi-Energy Industrial Systems[http://ecnu.summon.serialssolutions.com/#!/search?ho=t&l=zh-CN&q=Cooperation%20in%20Industrial%20Multi-agent%20Systems]��)
In \cite{Jennings91cooperationin,ferber1999multi,damavandi2016modeling,hanzo2009near}, plenty of researchers devote themselves to the research of multi-system and achieve great achievements.
% (Event-triggered distributed predictive control for the collaboration of multi-agent systems)
Zou et al.\cite{zou2016event} investigates  the problem of event-triggered distributed predictive control for multi-agent systems.
% (Consensus and Cooperation in Networked Multi-Agent Systems)
R.Olfati-Saber et al.\cite{4118472} provides a theoretical framework for the analysis of consensus algorithms for the decomposition of multi-agent networked system with an emphasis on the role if directed information flow, robustness to change in the network.
% (Decomposition Analysis of Indurstrial Power Demand of China based on Panel Data Model)
He et al.\cite{he2008decomposition} analyzes the decomposition under the industrial power demand.
% (iMRK: Demonstrator for Intelligent and Intuitive Human--Robot Collaboration in Industrial Manufacturing)
Dennis Mronga et al.\cite{deGeaFernandez2017} introduce the intelligent and intuitive dual-arm robotic system for industrial human-robot collaboration.
% (Developing collaboration mechanism for multi-agent systems with Petri net)
Fu-Shiung Hsieh \cite{DBLP:journals/eaai/Hsieh09} propose a collaboration mechanism of resource donation, including unilateral resource donation and reciprocal resource donation. During the process of our study, the collaboration algorithm basing on the premise of resource constraints. 


