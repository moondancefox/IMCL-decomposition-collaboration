\begin{abstract}
With the development of ``Industry 4.0'', the scale and complexity of industrial control system grow rapidly.
Hence, the analysis and verification of such systems face really big challenges.
Industry requires a reliable approach for decomposing the existing complex system model to multiple fine-grained and interactive models.
In this paper, we propose a general event-triggered language named \emph{IMCL} for modeling industrial control systems.
\emph{IMCL} can describe the physical resources and system in one unified model.
Specially, for the given resource constraints, we research the reliable and efficient approach of decomposition and collaboration based on \emph{IMCL} models to meet the industrial requirements. In particular, we have implemented the approach in our tool and get the encouraging results.

%With the \TODO{trend}{development} of ``Industry 4.0'', \TODO{the function and the intelligence of industrial system become sophisticated}{the scale and complexity of industrial system grows rapidly}.
%\TODO{Because there are more physical resources in the system, the industrial environment is more complicated.}{} Hence, the analysis and verification of such systems \TODO{have been much harder than before}{face really big challenges}.
%\TODO{It is a proper solution to decompose the complicated system model to multiple relatively simple models, and it is necessary to find a reliable way for the multiple models to collaborate with each other efficiently}{Industry requires a reliable approach for decomposing the existing complex system model to multiple fine-grained and interactive models.}.
%\TODO{In this paper, we introduce a general event-triggered language named \emph{IMCL} for modelling industrial systems.
%We show how \emph{IMCL} can describe the physical resources and system in one unified model.
%Within the given physical resource constraints, we present  reliable and efficient corresponding decomposition and collaboration algorithms based on \emph{IMCL} models. In particular, we have implemented these algorithms in a tool and get same satisfactory results.}{In this paper, we propose a general event-triggered language named \emph{IMCL} for modeling industrial systems.
%\emph{IMCL} describes the physical resources and system in one unified model.
%Following the given physical resource constraints, we present the reliable and efficient decomposition and collaboration algorithms based on \emph{IMCL} models to meet the industrial requirements. In particular, we have implemented these algorithms in a tool and get same encouraging results.}


%With the development of ``Industry 4.0'', the function and intelligence of industrial system become sophisticated.
%Because there are more physical resources in the system.
%The industrial environment is more complicated.
%Hence, the analysis and verification of such systems would be much harder than before.
%It is a proper solution to decompose the complicated system model to multiple Computing Units(CUs), and it is necessary to find a reliable way for the multiple CUs to collaborate with each other efficiently.
%In this paper, we propose a general event-triggered language named \emph{IMCL}, which can describe the physical resources and systems in one unified model.
%On the premise of the given physical resource constraints, we present the decomposition and collaboration algorithms that
%
%to ensure the consistency of the models during the
% that a collaborative multi-systems models can replace the complicated system model.
%On the premise of the given physical resource constraints, we propose the decomposition and collaboration algorithms based on the coordination of the system communication and collaboration functions. Ultimately, we can achieve the optimal solution that a collaborative multi-systems model can replace the traditional system model.


%\keywords{Program analysis, Industrial Control System, Model decomposition, Collaboration, Resource constraint}

\end{abstract}


% \keywords{Program analysis, Industrial Control System, Model decomposition, Collaboration, Resource constraint}


% It is a research focus to make use of the computing units(CUs) in the system, and it is a challenge to collaborate CUs with each other to partake the complicated system environment efficiently.

% ��һ�汾
% As the function and intelligence of industrial systems become more sophisticated, traditional single core processor will suffer more computing tasks. This paper aims to propose a method to decompose and collaborate one system under the physical resources constraint. Traditional single core processor is replaced by a cluster of heterogeneous and isomorphism computing units available in the system, enabling the replacement of a single model with multiple sub-models. In this paper, we propose a general event-triggered language named \emph{IMCL}, which can abstract the system resources and functions to achieve platform independence, and keeps the physical resources and systems in a unified model. On the premise of the given resource constraints, we propose the decomposition and collaboration algorithms based on the coordination of the system communication and collaboration functions, and ultimately to achieve the optimal solution that a collaborative multi-systems model can replace the traditional single-system model.

%\keywords{Program analysis, Model decomposition, Collaboration}
%\end{abstract}

% �ڶ��汾
%As the function and intelligence of industrial systems become sophisticated, the traditional industrial environment is becoming more complex and will suffer more computing tasks. It is a challenge to use multiple processors to collaborate with each other to partake the complex system environment efficiently. In this paper, we propose a method that the traditional processor can be replaced by a cluster of heterogeneous computing units available in one industrial system. To achieve platform independence of different systems, we present a general event-triggered language named \emph{IMCL}, which can describe the physical resources and systems in an unified model. On the premise of the given resource constraints, we propose the decomposition and collaboration algorithms based on the coordination of the system communication and collaboration functions, and ultimately to achieve the optimal solution that a collaborative multi-systems model can replace the traditional system model.

